\documentclass[xcolor=dvipsnames]{beamer}
\usepackage[T1]{fontenc}
\usepackage[utf8]{inputenc}
\usepackage{palatino}
\usepackage{beamerthemesplit}
\usefonttheme{serif}
\mode<presentation> { 
\usetheme{Szeged} 
\setbeamercovered{transparent} 
} 
\setbeamertemplate{footline}[frame number]
%\useoutertheme{infolines} 
\usecolortheme{dolphin} 
\usecolortheme{orchid}
\usecolortheme[named=MidnightBlue]{structure}

\usepackage{color}
\setbeamercolor{alerted text}{fg=OrangeRed}
\setbeamercovered{invisible}

\usepackage{graphicx}
\usepackage{amssymb}
\usepackage{epstopdf}
\usepackage{booktabs}
\usepackage{linguex}

\usepackage{tikz}
\usetikzlibrary{shapes.geometric,fit}

\usepackage{natbib}
\newcommand{\citeposs}[2][]{\citeauthor{#2}'s (\citeyear[#1]{#2})}
\bibpunct[:]{(}{)}{;}{a}{}{,}


\def\highlight#1{\textbf{#1}}
\def\foc{$_{\textnormal{\textsc{f}}}$}
\def\bad{\leavevmode\llap{*}}
\def\jg#1{\leavevmode\llap{#1}}
\def\questionable{\leavevmode\llap{$^\textnormal{?}$}}


%\usepackage{amssymb}
\usepackage{MnSymbol}
\usepackage{stmaryrd}

\def\>{\rangle}
\def\<{\langle}
\newcommand{\valueof}[1]{{\ensuremath{\llbracket #1 \rrbracket}}} % for double semantic brackets
\newcommand{\tvalueof}[1]{$\llbracket$#1$\rrbracket$}

\def\sk#1{\setbox1\hbox{#1}\rule[.5ex]{\wd1}{.4pt}\hspace*{-\wd1}{\box1}}

\def\ldot{\,.\,}

\usepackage{skak}
\def\undef{\rm{\mate}}
\def\undefe{\ensuremath{{\rm{\mate}}_e}}
\def\trueval{{\rm{T}}}
\def\falseval{{\rm{F}}}
\def\undeft{\ensuremath{{\rm{\mate}}}}
\def\undefconstant{\ensuremath{\star}}
\def\subindivid{\ensuremath{\sqsubseteq_e}}
\def\bigjoin{\ensuremath{\bigsqcup}}
\def\join{\ensuremath{\sqcup}}
\def\bigmeet{\ensuremath{\bigsqcap}}
\def\meet{\ensuremath{\sqcap}}
\def\subind{\ensuremath{\preccurlyeq}}
\def\propersubind{\ensuremath{\prec}}
\def\tvrank{\ensuremath{\sqsubseteq_t}}
\def\cum{\textsc{*}}
\def\msum{\ensuremath{\oplus}}
\def\const#1{\textsc{#1}}
%% gloss.tex
% macros for glosses and f-structures

\newenvironment{gloss}
{\begin{tabular}[t]{*{20}{@{\hspace{0pt}}l@{~}}}}
  {\end{tabular}}
\newcommand{\trans}[1]{\multicolumn{20}{@{\hspace{0pt}}l}{`#1'}}

\def\cat#1{$_\textnormal{\sc #1}$}
\def\vset#1{$\left\{{#1}\right\}$}

%features

\newcommand{\sg}{\textnormal{{\sc sg}}}
\newcommand{\pl}{\textnormal{{\sc pl}}}
\newcommand{\du}{\textnormal{{\sc du}}}

\newcommand{\fem}{{\sc fem}}
\newcommand{\masc}{{\sc masc}}

\newcommand{\nom}{{\sc nom}}
\newcommand{\dat}{{\sc dat}}
\newcommand{\acc}{{\sc acc}}
\newcommand{\gen}{{\sc gen}}
\newcommand{\erg}{{\sc erg}}
\newcommand{\abs}{{\sc abs}}
\newcommand{\instr}{\sc instr}
\newcommand{\poss}{{\sc poss}}
\newcommand{\caus}{{\sc caus}}
\newcommand{\loc}{{\sc loc}}

\newcommand{\defn}{{\sc def}}
\newcommand{\indef}{{\sc indef}}

\newcommand{\past}{{\sc past}}
\newcommand{\pres}{{\sc pres}}
\newcommand{\perf}{{\sc perf}}
\newcommand{\fut}{{\sc fut}}
\newcommand{\pot}{{\sc pot}}
\newcommand{\cnd}{{\sc cond}}
\newcommand{\sbj}{{\sc sbj}}

\newcommand{\infin}{{\sc inf}}

\newcommand{\cont}{{\sc cont}} % continuous aspect
\newcommand{\fv}{\sc fv} % final vowel (Bantu, indicative mood marker?)

\newcommand{\cl}{{\sc cl}} %class

\newcommand{\supr}{{\sc sup}}
\newcommand{\compr}{\textsc{comp}}

\newcommand{\wk}{{\sc w}}

\newcommand{\utr}{{\sc utr}}
\newcommand{\neu}{{\sc neu}}

\def\root{\raisebox{1ex}{$\sqrt{}$}}

\newcommand{\short}{\textsc{short}}



% attributes

\newcommand{\num}{{\sc num}}
\newcommand{\pers}{{\sc pers}}
\newcommand{\gend}{{\sc gend}}
%\newcommand{\case}{{\sc case}}
\newcommand{\tense}{{\sc tense}}

\newcommand{\pred}{{\sc pred}}
\newcommand{\ind}{{\sc index}}

\newcommand{\subj}{{\sc subj}}
\newcommand{\obj}{{\sc obj}}
\newcommand{\xcomp}{{\sc xcomp}}
\newcommand{\comp}{{\sc comp}}
\newcommand{\focus}{{\sc focus}}


% arrows

\newcommand{\up}[1]{$\uparrow${\sc #1}}
\newcommand{\dn}[1]{$\downarrow${\sc #1}}
\newcommand{\upisdn}{\up{}=\dn{}}


\def\bad{\leavevmode\llap{*}}
\def\jg#1{\leavevmode\llap{#1}}
\def\sk#1{\setbox1\hbox{#1}\rule[.5ex]{\wd1}{.4pt}\hspace*{-\wd1}{\box1}}
\newcommand{\myalt}[2]{$\left\{\begin{array}{l}\textnormal{#1}\\\textnormal{#2}\end{array}\right\}$}
\usepackage{qtree}


\title{Adverbial Superlatives}
\author[Comparatives Group]{Liz, Tim, Greg, and Andreea}
\institute{SIASSI Summer Institute 2015}
%\date{Sinn und Bedeutung 2014}
\begin{document}

\frame{\titlepage}



\begin{frame}{Adverbial superlatives}

\ex. John ran (the) fastest.

Questions:
\begin{itemize}
\item Why can we have {\em the} (in English) (but not German)? 
\item Is it freely insertable and if so why?
\item Do adverbial superlatives have both absolute and relative
  readings?
\item How to analyze the elements involved? Do adverbial superlatives
  help to decide between analyses of adjectival superlatives?
\end{itemize}
\end{frame}

\begin{frame}{Previous work}
\citet{krasikova:2012}:

\begin{itemize}
\item ``One of the two remaining issues to be addressed is the
  optionality of {\em the} in adverbial and predicative
  superlatives.'' Predicative superlative:

\ex. John is (the) smartest.

\item ``Under the present analysis, both an absolute and a
  comparative construal are possible in such cases and derive the same
  truth conditions.''
\end{itemize}

\end{frame}


\begin{frame}{Syntactic restrictions}

\ex.
\a. John carefully opened the window.
\b. John opened the window carefully.

\ex.
\a. John ran (the) fastest.
\b. *John (the) fastest drove.

\end{frame}


\begin{frame}{Focus-sensitivity}


\ex. John$_F$ ran the fastest on Tuesday.

\ex. John ran the fastest on Tuesday$_F$.

\end{frame}


\begin{frame}{Cf.\ relative readings of adnominal superlatives}

\ex. {Of her 3 sisters, [Jean]$_F$ bought the most expensive
  book. \label{deg-rel}}

\ex. {There was the least snow on [Tuesday]$_F$. \label{snow}}

\ex. {Who received the fewest letters? \label{wh}}

\ex. {The one who receives the most votes wins. \label{rel}}

\ex. {A: How do you win this contest?\\B: By [ PRO putting the tallest
  plant on the table ]. \label{pro}}

\end{frame}



\begin{frame}{Anatomy of a relative reading}

\begin{center}
\begin{tikzpicture}
%put some nodes on the left
\foreach \x in {1,2,3}{
\node[fill,blue,circle,inner sep=2pt] (d\x) at (0,\x) {};
}
\node[fit=(d1) (d2) (d3),ellipse,draw,minimum width=1cm,label={[label
  distance=.3cm,blue]90:contrast set},label={[label
  distance=.6cm,purple]270:association relation},draw=blue] {}; 
%put some nodes on the center
\foreach \x[count=\xi] in {0.5,1.5,...,4}{
\node[fill,ForestGreen,circle,inner sep=2pt] (r\xi) at (2,\x) {};
}

\node[fit=(r1) (r2) (r3) (r4),ellipse,draw,minimum width=1.5cm,draw=ForestGreen,label={[label
  distance=.3cm,ForestGreen]90:measured entities},label={[label
  distance=.3cm,orange]280:measure}] {}; 

%put some nodes on the right
\foreach \x[count=\xi] in {0.75,1.5,...,3}{
\node[fill,red,circle,inner sep=2pt] (c\xi) at (4,\x) {};
}
\node[fit=(c1) (c2) (c3) (c4) ,ellipse,draw,minimum width=1.5cm,label={[label
  distance=.3cm,red]90:degrees},draw=red] {};
\draw[-latex,purple] (d1) -- (r1);
\draw[-latex,purple] (d1) -- (r2);
\draw[-latex,purple] (d2) -- (r2);
\draw[-latex,purple] (d3) -- (r3);
\draw[-latex,purple] (d3) -- (r4);


\draw[-latex,orange] (r1) -- (c2);
\draw[-latex,orange] (r2) -- (c3);
\draw[-latex,orange] (r3) -- (c3);
\draw[-latex,orange] (r4) -- (c4);


\end{tikzpicture}
\end{center}

\end{frame}

\begin{frame}{\textcolor{blue}{John}$_F$ \textcolor{purple}{climbed}
    the \textcolor{orange}{high}est \textcolor{ForestGreen}{mountain}:
  Relative reading}

\begin{center}
\begin{tikzpicture}
%put some nodes on the left
\foreach \x in {1,2,3}{
\node[fill,blue,circle,inner sep=2pt] (d\x) at (0,\x) {};
}
\node[fit=(d1) (d2) (d3),ellipse,draw,minimum width=1cm,label={[label
  distance=.3cm,blue]90:climbers},label={[label
  distance=.6cm,purple]270:$x$ climbed $y$},draw=blue] {}; 
%put some nodes on the center
\foreach \x[count=\xi] in {0.5,1.5,...,4}{
\node[fill,ForestGreen,circle,inner sep=2pt] (r\xi) at (2,\x) {};
}

\node[fit=(r1) (r2) (r3) (r4),ellipse,draw,minimum width=1.5cm,draw=ForestGreen,label={[label
  distance=.3cm,ForestGreen]90:climbed mountains},label={[label
  distance=.3cm,orange]280:$y$ is $d$-high}] {}; 

%put some nodes on the right
\foreach \x[count=\xi] in {0.75,1.5,...,3}{
\node[fill,red,circle,inner sep=2pt] (c\xi) at (4,\x) {};
}
\node[fit=(c1) (c2) (c3) (c4) ,ellipse,draw,minimum width=1.5cm,label={[label
  distance=.3cm,red]90:degrees},draw=red] {};
\draw[-latex,purple] (d1) -- (r1);
\draw[-latex,purple] (d1) -- (r2);
\draw[-latex,purple] (d2) -- (r2);
\draw[-latex,purple] (d3) -- (r3);
\draw[-latex,purple] (d3) -- (r4);


\draw[-latex,orange] (r1) -- (c2);
\draw[-latex,orange] (r2) -- (c3);
\draw[-latex,orange] (r3) -- (c3);
\draw[-latex,orange] (r4) -- (c4);


\end{tikzpicture}
\end{center}

\end{frame}



\begin{frame}{\textcolor{purple}{John climbed}
    the \textcolor{orange}{high}est \textcolor{ForestGreen}{mountain}
    \textcolor{purple}{on} \textcolor{blue}{Tuesday}}

\begin{center}
\begin{tikzpicture}
%put some nodes on the left
\foreach \x in {1,2,3}{
\node[fill,blue,circle,inner sep=2pt] (d\x) at (0,\x) {};
}
\node[fit=(d1) (d2) (d3),ellipse,draw,minimum width=1cm,label={[label
  distance=.3cm,blue]90:days},label={[label
  distance=.6cm,purple]270:John climbed $y$ on $x$},draw=blue] {}; 
%put some nodes on the center
\foreach \x[count=\xi] in {0.5,1.5,...,4}{
\node[fill,ForestGreen,circle,inner sep=2pt] (r\xi) at (2,\x) {};
}

\node[fit=(r1) (r2) (r3) (r4),ellipse,draw,minimum width=1.5cm,draw=ForestGreen,label={[label
  distance=.3cm,ForestGreen]90:climbed mountains},label={[label
  distance=.3cm,orange]280:$y$ is $d$-high}] {}; 

%put some nodes on the right
\foreach \x[count=\xi] in {0.75,1.5,...,3}{
\node[fill,red,circle,inner sep=2pt] (c\xi) at (4,\x) {};
}
\node[fit=(c1) (c2) (c3) (c4) ,ellipse,draw,minimum width=1.5cm,label={[label
  distance=.3cm,red]90:degrees},draw=red] {};
\draw[-latex,purple] (d1) -- (r1);
\draw[-latex,purple] (d1) -- (r2);
\draw[-latex,purple] (d2) -- (r2);
\draw[-latex,purple] (d3) -- (r3);
\draw[-latex,purple] (d3) -- (r4);


\draw[-latex,orange] (r1) -- (c2);
\draw[-latex,orange] (r2) -- (c3);
\draw[-latex,orange] (r3) -- (c3);
\draw[-latex,orange] (r4) -- (c4);


\end{tikzpicture}
\end{center}

\end{frame}

\begin{frame}{Recipe for a relative reading (requires focus)}
\begin{itemize}
\item The \textcolor{orange}{measure} is whatever the superlative
  morpheme attaches to.
\item The \textcolor{purple}{association relation} can be derived by
abstracting over the focus and the phrase containing the both the superlative
expression and the property it modifies.
\item The \textcolor{blue}{contrast set} is (a contextually restricted
  subset of) the focus alternatives.
\item The \textcolor{ForestGreen}{measured entities} are the members of the range of the
  association relation (the set of $y$s such that $xRy$). They are
  drawn from a \textcolor{ForestGreen}{domain} given by the phrase the
  superlative expression modifies.
\end{itemize}
\end{frame}


\begin{frame}{\textcolor{blue}{John}$_F$ \textcolor{ForestGreen}{drove}
    the \textcolor{orange}{fast}est:
  Relative reading}

\begin{center}
\begin{tikzpicture}
%put some nodes on the left
\foreach \x in {1,2,3}{
\node[fill,blue,circle,inner sep=2pt] (d\x) at (0,\x) {};
}
\node[fit=(d1) (d2) (d3),ellipse,draw,minimum width=1cm,label={[label
  distance=.3cm,blue]90:drivers},label={[label
  distance=.6cm,purple]270:$x$ is the agent of $y$},draw=blue] {}; 
%put some nodes on the center
\foreach \x[count=\xi] in {0.5,1.5,...,4}{
\node[fill,ForestGreen,circle,inner sep=2pt] (r\xi) at (2,\x) {};
}

\node[fit=(r1) (r2) (r3) (r4),ellipse,draw,minimum width=1.5cm,draw=ForestGreen,label={[label
  distance=.3cm,ForestGreen]90:driving events},label={[label
  distance=.3cm,orange]280:$y$ is $d$-fast}] {}; 

%put some nodes on the right
\foreach \x[count=\xi] in {0.75,1.5,...,3}{
\node[fill,red,circle,inner sep=2pt] (c\xi) at (4,\x) {};
}
\node[fit=(c1) (c2) (c3) (c4) ,ellipse,draw,minimum width=1.5cm,label={[label
  distance=.3cm,red]90:degrees},draw=red] {};
\draw[-latex,purple] (d1) -- (r1);
\draw[-latex,purple] (d1) -- (r2);
\draw[-latex,purple] (d2) -- (r2);
\draw[-latex,purple] (d3) -- (r3);
\draw[-latex,purple] (d3) -- (r4);


\draw[-latex,orange] (r1) -- (c2);
\draw[-latex,orange] (r2) -- (c3);
\draw[-latex,orange] (r3) -- (c3);
\draw[-latex,orange] (r4) -- (c4);


\end{tikzpicture}
\end{center}

\end{frame}




\begin{frame}{John climbed the highest mountain: Absolute reading}

\begin{center}
\begin{tikzpicture}
%put some nodes on the left
\foreach \x in {1,2,3}{
\node[fill,blue,circle,inner sep=2pt] (d\x) at (0,\x) {};
}
\node[fit=(d1) (d2) (d3),ellipse,draw,minimum width=1cm,label={[label
  distance=.3cm,blue]90:mountains},label={[label
  distance=.6cm,purple]280:$x=y$},draw=blue] {}; 
%put some nodes on the center

\foreach \x in {1,2,3}{
\node[fill,ForestGreen,circle,inner sep=2pt] (r\x) at (2,\x) {};
}
\node[fit=(r1) (r2) (r3) ,ellipse,draw,minimum width=1.5cm,draw=ForestGreen,label={[label
  distance=.3cm,ForestGreen]90:mountains},label={[label
  distance=.3cm,orange]276:$y$ is $d$-high}] {}; 

%put some nodes on the right
\foreach \x[count=\xi] in {0.75,1.5,...,3}{
\node[fill,red,circle,inner sep=2pt] (c\xi) at (4,\x) {};
}
\node[fit=(c1) (c2) (c3) (c4) ,ellipse,draw,minimum width=1.5cm,label={[label
  distance=.3cm,red]90:degrees},draw=red] {};
\draw[-latex,purple] (d1) -- (r1);
\draw[-latex,purple] (d2) -- (r2);
\draw[-latex,purple] (d3) -- (r3);


\draw[-latex,orange] (r1) -- (c2);
\draw[-latex,orange] (r2) -- (c3);
\draw[-latex,orange] (r3) -- (c3);
%\draw[-latex,orange] (r4) -- (c4);


\end{tikzpicture}
\end{center}

\end{frame}


%\node[fit=(r1) (r2) (r3) (r4),ellipse,draw,minimum width=1.5cm},draw=ForestGreen] {}; 


\begin{frame}{Recipe for an absolute reading}
\begin{itemize}
\item The \textcolor{orange}{measure} is whatever the superlative
  morpheme attaches to.
\item The \textcolor{purple}{association relation} is equality.
\item The \textcolor{blue}{contrast set} collapses with the set of
  measured entities.
\item The \textcolor{ForestGreen}{measured entities} are
  drawn from a \textcolor{ForestGreen}{domain} given by the phrase the
  superlative expression modifies.
\end{itemize}
\end{frame}


\begin{frame}{Do adverbial superlatives have absolute readings?}

\ex. John ran the fastest.

Two imaginable absolute readings:
\begin{itemize}
\item John was the agent of running event $e$, and $e$ had a speed
  that was greater than all other speeds in the context.
\item John was the agent of running event $e$, and $e$ was faster than
  all other running events in the context.
\end{itemize}
\end{frame}


\begin{frame}{John drove the fastest: Absolute reading (I)}

\begin{center}
\begin{tikzpicture}
%put some nodes on the left
\foreach \x in {1,2,3}{
\node[fill,blue,circle,inner sep=2pt] (d\x) at (0,\x) {};
}
\node[fit=(d1) (d2) (d3),ellipse,draw,minimum width=1cm,label={[label
  distance=.3cm,blue]90:speeds},label={[label
  distance=.6cm,purple]280:$x=y$},draw=blue] {}; 
%put some nodes on the center

\foreach \x in {1,2,3}{
\node[fill,ForestGreen,circle,inner sep=2pt] (r\x) at (2,\x) {};
}
\node[fit=(r1) (r2) (r3) ,ellipse,draw,minimum width=1.5cm,draw=ForestGreen,label={[label
  distance=.3cm,ForestGreen]90:speeds},label={[label
  distance=.3cm,orange]276:$y=d$}] {}; 

%put some nodes on the right
\foreach \x[count=\xi] in {0.75,1.5,...,3}{
\node[fill,red,circle,inner sep=2pt] (c\xi) at (4,\x) {};
}
\node[fit=(c1) (c2) (c3) (c4) ,ellipse,draw,minimum width=1.5cm,label={[label
  distance=.3cm,red]90:degrees},draw=red] {};
\draw[-latex,purple] (d1) -- (r1);
\draw[-latex,purple] (d2) -- (r2);
\draw[-latex,purple] (d3) -- (r3);


\draw[-latex,orange] (r1) -- (c2);
\draw[-latex,orange] (r2) -- (c3);
\draw[-latex,orange] (r3) -- (c4);
%\draw[-latex,orange] (r4) -- (c4);


\end{tikzpicture}
\end{center}

\end{frame}


\begin{frame}{Private driving tutor scenario}

  John has a private driving tutor. The tutor says that John can
  choose to drive 40mph, 50mph, or 60mph. John was unsure which to
  choose, but in the end...

\ex.
\a. {...he drove the fastest speed.\label{tutor-the-fastest-speed}}
\b. {*...he drove fastest.\label{tutor-fastest}}
\c. {?...he drove the fastest.\label{tutor-the-fastest}}
\d. {...of the given speeds, he drove the
    fastest.\label{tutor-the-fastest-of}}

\end{frame}


\begin{frame}{John drove the fastest: Absolute reading (II)}

\begin{center}
\begin{tikzpicture}
%put some nodes on the left
\foreach \x in {1,2,3}{
\node[fill,blue,circle,inner sep=2pt] (d\x) at (0,\x) {};
}
\node[fit=(d1) (d2) (d3),ellipse,draw,minimum width=1cm,label={[label
  distance=.3cm,blue]90:drivings},label={[label
  distance=.6cm,purple]280:$x=y$},draw=blue] {}; 
%put some nodes on the center

\foreach \x in {1,2,3}{
\node[fill,ForestGreen,circle,inner sep=2pt] (r\x) at (2,\x) {};
}
\node[fit=(r1) (r2) (r3) ,ellipse,draw,minimum width=1.5cm,draw=ForestGreen,label={[label
  distance=.3cm,ForestGreen]90:drivings},label={[label
  distance=.3cm,orange]276:$y$ is $d$-fast}] {}; 

%put some nodes on the right
\foreach \x[count=\xi] in {0.75,1.5,...,3}{
\node[fill,red,circle,inner sep=2pt] (c\xi) at (4,\x) {};
}
\node[fit=(c1) (c2) (c3) (c4) ,ellipse,draw,minimum width=1.5cm,label={[label
  distance=.3cm,red]90:degrees},draw=red] {};
\draw[-latex,purple] (d1) -- (r1);
\draw[-latex,purple] (d2) -- (r2);
\draw[-latex,purple] (d3) -- (r3);


\draw[-latex,orange] (r1) -- (c2);
\draw[-latex,orange] (r2) -- (c3);
\draw[-latex,orange] (r3) -- (c3);
%\draw[-latex,orange] (r4) -- (c4);


\end{tikzpicture}
\end{center}

\end{frame}


\begin{frame}{Driving practice scenario}

John practiced driving on the slippery road three times today. The
first time he drove 30mph, the second time he drove 40mph, and the
third time he drove 50mph.

\ex. \#John drove fastest.

There must be another driver (or we're comparing driving to other
modes of transportation).

\end{frame}

\begin{frame}{Where we are}

\begin{itemize}
\item Adverbial superlatives do not have absolute readings.
\item Why? Perhaps the contrast set must consist of suitably
  individuable entities. Events are not suitably individuable.
\item Question: Do predicative superlatives have absolute readings? Do
  they have relative readings?
\end{itemize}

\end{frame}



%\begin{frame}
\small
\bibliographystyle{sp}
\bibliography{master}
%\end{frame}

\end{document}