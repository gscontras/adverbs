\documentclass{article}
\def\highlight#1{\textbf{#1}}
\def\foc{$_{\textnormal{\textsc{f}}}$}
\def\bad{\leavevmode\llap{*}}
\def\jg#1{\leavevmode\llap{#1}}
\def\questionable{\leavevmode\llap{$^\textnormal{?}$}}


%\usepackage{amssymb}
\usepackage{MnSymbol}
\usepackage{stmaryrd}

\def\>{\rangle}
\def\<{\langle}
\newcommand{\valueof}[1]{{\ensuremath{\llbracket #1 \rrbracket}}} % for double semantic brackets
\newcommand{\tvalueof}[1]{$\llbracket$#1$\rrbracket$}

\def\sk#1{\setbox1\hbox{#1}\rule[.5ex]{\wd1}{.4pt}\hspace*{-\wd1}{\box1}}

\def\ldot{\,.\,}

\usepackage{skak}
\def\undef{\rm{\mate}}
\def\undefe{\ensuremath{{\rm{\mate}}_e}}
\def\trueval{{\rm{T}}}
\def\falseval{{\rm{F}}}
\def\undeft{\ensuremath{{\rm{\mate}}}}
\def\undefconstant{\ensuremath{\star}}
\def\subindivid{\ensuremath{\sqsubseteq_e}}
\def\bigjoin{\ensuremath{\bigsqcup}}
\def\join{\ensuremath{\sqcup}}
\def\bigmeet{\ensuremath{\bigsqcap}}
\def\meet{\ensuremath{\sqcap}}
\def\subind{\ensuremath{\preccurlyeq}}
\def\propersubind{\ensuremath{\prec}}
\def\tvrank{\ensuremath{\sqsubseteq_t}}
\def\cum{\textsc{*}}
\def\msum{\ensuremath{\oplus}}
\def\const#1{\textsc{#1}}
\makeatletter
     
\edef\@backslash{\expandafter\@car\string\relax\@nil}
     
\def\@warnredef#1{\typeout{WARNING: The `#1' control sequence is being
 redefined.}}
     
\def\@warnnoredef#1{\typeout{WARNING: Aborting redefinition of the `#1' control
 sequence.}}
     
\def\@ifcsundefined#1#2#3{\ifx #1\anUndefinedMacro
   \def\@@temp{#2}\else\ifx #1\relax\def\@@temp{#2}\else
   \def\@@temp{#3}\fi\fi\@@temp}
     
% Change LaTex behavior.
     
\def\@notdefinable{\@warnredef{\@backslash\@tempa}}
     
% Define \redefcheck{\ctlseq} that just checks the sequence
% and ignores it.
     
\def\redefcheck#1{\@ifcsundefined#1{}{\@warnredef{\string#1}}}
     
% \redefabort{\oneargdefiner}{\ctlseq} puts out \oneargdefiner\ctlseq,
% except that if \ctlseq is already defined, it just gobbles both
% of them.
     
\def\redefabort#1#2{\@ifcsundefined#2{#1#2}{\@warnnoredef{\string#2}}}
     
% Define \newdef = \def with \redefcheck, and similarly for \gdef and \let.
     
\def\newdef#1{\redefcheck{#1}\def#1}
\def\newgdef#1{\redefcheck{#1}\gdef#1}
\def\newlet#1{\redefcheck{#1}\let#1}
     
% Add redefinition checking to \newif.
% \newif is declared as \outer, so we need to delay the definition
% of \latexnewif until we've scanned the definition of the new
% \newif.
     
\let\@tempa\newif
\def\newif#1{\redefcheck{#1}\latexnewif#1}
\let\latexnewif\@tempa
     
% Now for the allocation commands, which I'm setting up to skip the
% redefinition if their target control sequences are already defined.
% These are also declared \outer, but the above \@tempa trick won't work
% here; it's necessary to use another level of macros.
     
\def\latexnewbox{\latexouternewbox}
\let\latexouternewbox\newbox
\def\newbox#1{\redefabort\latexnewbox{#1}}
     
\def\latexnewcount{\latexouternewcount}
\let\latexouternewcount\newcount
\def\newcount#1{\redefabort\latexnewcount{#1}}
     
\def\latexnewdimen{\latexouternewdimen}
\let\latexouternewdimen\newdimen
\def\newdimen#1{\redefabort\latexnewdimen{#1}}
     
\def\latexnewtoks{\latexouternewtoks}
\let\latexouternewtoks\newtoks
\def\newtoks#1{\redefabort\latexnewtoks{#1}}
     
\makeatother
% end redefine
    
% A referenceable "sentences" list environment for Latex.  Remember to
% put "{}" % after the \item command if the sentence begins with "[".
% There is also a "subsentences" environment.
     
\makeatletter
     
\newif\ifinsentences
\insentencesfalse
     
\newcounter{sentence}
\newcounter{subsentence}[sentence]
     
% These next definitions define the format of the "sentence"
% and "subsentence" counters.  It is tricky to get them right.
% First of all, the parentheses in "(4)" and "(2iii)" are
% defined to be part of the printed representation of the
% "sentence" and "subsentence" counters.  This is so that
% parentheses will be included in \ref - \label constructions.
% Second, the "subsentence" counter must print out with the
% "sentence" number included as well; this is similar to
% what happens with the section and chapter counters.
% Third, despite the second requirement, the "subsentence"
% counter must appear without the sentence number when it
% is actually used to number a subsentence.
%
% We use two auxiliary control sequences here and below in order
% to make it easier to change the subsentence numbering style.
% The two control sequences are \subsentencenumbering, which should
% be \let to something like \roman or \alph, and
% \subsentencepunctuation, which controls the format of subsentence
% numbers as they appear before the subsentences.
     
\def\thesentence{(\arabic{sentence})}
\def\thesubsentence{(\arabic{sentence}\subsentencenumbering{subsentence})}
     
% These next two definitions are defaults.
     
\let\subsentencenumbering\alph
\def\subsentencepunctuation#1{#1.}
     
%%% Using \begin{list} and \end{list} in the argument of a
%%% \newenvironment command doesn't work properly; I have to call
%%% \list and \endlist directly.
     
\newenvironment{sentences}{\list{\thesentence}%
{\edef\sent@nbr{\the\c@sentence}\usecounter{sentence}%
\setcounter{sentence}{\sent@nbr}%
\ifinsentences \errmessage{Nested sentences environment.} \fi
\def\makelabel##1{##1\hfil}%
% the following command prevents indention problems when the number of
% sentences in the document falls in the set {99,...,999}
% (R. E. Wargaski Jr., 31 Oct 89)
\setlength{\leftmargin}{1.0cm}
\leftmargin\leftmargin\labelwidth\leftmargin
  \advance\labelwidth-\labelsep
\insentencestrue}}{\endlist}
     
\newenvironment{subsentences}{\list
{\subsentencepunctuation{\subsentencenumbering{subsentence}}}%
{\edef\sent@nbr{\the\c@subsentence}\usecounter{subsentence}%
\setcounter{subsentence}{\sent@nbr}%
% the following command prevents indention problems when the number of
% sentences in the document falls in the set {99,...,999}
% (R. E. Wargaski Jr., 31 Oct 89)
\setlength{\leftmargin}{1.0cm}
\setlength{\rightmargin}{\leftmargin}
\leftmargin\leftmargin\labelwidth\leftmargin
  \advance\labelwidth-\labelsep
\def\makelabel##1{##1\hfil}%
\ifinsentences\relax\else
   \errmessage{\string\begin{subsentences} not in sentences environment}\fi}}%
{\endlist}
\newcommand{\sentence}[1]{\begin{sentences}\item{#1}\end{sentences}}
     
\newdef\nextsentence{\@ifnextchar[{\@argnextsentence}{\@argnextsentence[1]}%]
}
     
\newdef\@argnextsentence[#1]{\bgroup
  \advance\c@sentence by #1\relax
  \thesentence
  \egroup}
     
\newdef\nextsentencesub{\@ifnextchar[{\@argnextsentencesub
   }{\@argnextsentencesub[1]}%]
}
     
\newdef\@argnextsentencesub[#1]#2{\bgroup
  \advance\c@sentence by #1\relax
  \c@subsentence=#2\relax
  \thesubsentence
  \egroup}


\begin{document}

\section{Anatomy of a relative reading}

\sentence{John$_F$ climbed the highest mountain. [relative]}

\begin{itemize}
\item Contrast set: John and his focus-alternatives
\item Association relation: $x$ climbed (mountain) $y$
\item Measured entities: mountain 1, mountain 2, etc.
\item Measure: $y$ is $d$-high 
\item Domain of measured entities: mountain
\end{itemize}

\sentence{John climbed the highest mountain on Tuesday$_F$ [relative]}
\begin{itemize}
\item Contrast set: Tuesday and its focus alternatives (days)
\item Association relation: John climbed (mountain) $y$ on $x$
\item Measured entities: mountain 1, mountain 2, etc.
\item Measure: $y$ is $d$-high
\item Domain of measured entities: mountain
\end{itemize}


\sentence{John climbed the highest mountain. [absolute]}

\begin{itemize}
\item Contrast set: mountain 1, mountain 2, etc.
\item Association relation: $x=y$
\item Measured entities: mountain 1, mountain 2, etc.
\item Measure: $y$ is $d$-high
\item Domain of measured entities: mountain
\end{itemize}


In general, elements of the contrast set are related to elements of
the measured entities via the association relation, and the measured
entities are members of the domain.

Suppose you are a computer asked to identify the pieces of a sentence
containing a superlative. Then you can follow the following procedure.
If there is no focus, then you have an absolute reading:
\begin{itemize}
\item The association relation is identity.
\item The contrast set is given by the head noun.
%the set of measured entities.
\end{itemize}
In the presence of focus,  you can have a relative or an absolute
reading. For the relative reading:
\begin{itemize}
\item The association relation can be derived by
abstracting over the focus and the noun phrase (more generally:
smallest maximal projection) containing the superlative
phrase. %cf. Farkas & Kiss argument against contextually given assoc relns
\item The contrast set is (a contextually restricted
  subset of) the focus alternatives.
\item The measured entities are the members of the range of the
  association relation (the set of $y$s such that $xRy$).
\end{itemize}
For the absolute reading: follow the procedure for the case where
there is no focus.


% to interpret a superlative sentence: 

% first figure out ass'n rel'n via focus;
% that gives you c.s. + m.e.; (by abstracting over their positions -
% focus for the contrast set, and superlative phrase for the m.e.) 
%      BUT if there is no focus, then a.r. defaults to =;
%      = can't give you c.s. or m.e.; therefore
%            you default to the NOUN's extension (some subset thereof)
%            to give you both c.s. + m.e. and you then get an absolute
%            reading;
%            BUT if you cannot default to the noun b/c there is no noun
%            (as is the case w adverbials and predicative),
%                 THEN you cannot use = as a.r.; 
%                 AND therefore cannot get an absolute reading
           





\section{Adverbial superlatives}

\paragraph{Question:} Do adverbial superlatives have both absolute and relative readings?

\paragraph{Answer:} They certainly seem have relative readings.

\sentence{John$_F$ drove (the) fastest.\label{drove-the-fastest}}

This looks like a relative reading either like this (\textbf{degree hypothesis}):
\begin{itemize}
\item Contrast set: John and his focus-alternatives.
\item Association relation: $x$ ran $y$-fast
\item Measured entities: running speed 1, running speed 2, etc.
\item Measure: $y$ is $d$(-great)
\item Domain of measured entities: speeds
\end{itemize}
or like this (\textbf{event hypothesis}):
\begin{itemize}
\item Contrast set: John and his focus-alternatives
\item Association relation: $x$ is the agent of $y$
\item Measured entities: running event 1, running event 2, etc.
\item Measure: $y$ is $d$-fast
\item Domain of measured entities: running events
\end{itemize}
What would an absolute reading look like? The association relation
would be identity, so the contrast set is identical to the set of
measured entities. 
\begin{itemize}
\item \textbf{Absolute reading (degree hypothesis)}\\
John drove $y$-fast, and $y$ is greater in magnitude than all other
speeds in the context.
\item \textbf{Absolute reading (event hypothesis)}\\
John was the agent of driving event $y$, and $y$ is faster than all
other driving events in the context.
\end{itemize}
Prediction of degree hypothesis: \ref{tutor-fastest} should be
possible, and mean the same thing as \ref{tutor-the-fastest-speed}.

\sentence{John has a private driving tutor. The tutor says that John
  can choose to drive 40mph, 50mph, or 60mph. John was unsure which
  to choose, but in the end...
\begin{subsentences}
\item{...he drove the fastest speed.\label{tutor-the-fastest-speed}}
\item{*...he drove fastest.\label{tutor-fastest}}
\item{?...he drove the fastest.\label{tutor-the-fastest}}
\item{...of the given speeds, he drove the
    fastest.\label{tutor-the-fastest-of}}
\end{subsentences}}

Another prediction of the degree hypothesis, assuming an absolute
reading cannot independently be ruled out, is that \ref{j+s-fastest}
should have a non-contradictory reading, where John and Susie both
drove the fastest of the contextually available speed.
%
\sentence{
\begin{subsentences}
\item{*John drove fastest. Susie did too.\label{j+s-fastest}}
\item{?John drove the fastest. Susie did too.\label{j+s-the-fastest}}
\item{Of all the speeds, John drove *(the) fastest. Susie did too.}
\end{subsentences}}

The event hypothesis does not predict \ref{tutor-fastest} to be
felicitous on an absolute reading, because there is only one driving
events in the context.

Compositional derivation:

\sentence{\textit{fast} $\leadsto \lambda e \lambda d \ldot \const{fast}(e,d)$}
\sentence{\textit{-est}$_\theta$$\leadsto \lambda C_{\<e,t\>}\lambda
  G_{\<d,\<e,t\>\>}\lambda x_\tau$\\$ \exists d [ G(x,d) \land \forall x'
  [[C(x') \land x'\neq x] \rightarrow \neg G(x',d)] ]$\\
`there is a degree to which $x$ is $G$ and no distinct $x'$ in $C$ is $G$'}

\sentence{\textit{fastest}$_C$ $\leadsto \lambda x \ldot \exists d [ \const{fast}(x,d) \land \forall x'
  [[C(x') \land x'\neq x] \rightarrow \neg \textsc{fast}(x',d)] ] $}

\sentence{\textit{drove} $\leadsto \lambda e \ldot \const{driving}(e)$}

\sentence{\textit{drove fastest} $\leadsto \lambda x \ldot
  \const{driving}(x) \land  \exists d [ \const{fast}(x,d) \land \forall x'
  [[C(x') \land x'\neq x] \rightarrow \neg \textsc{fast}(x',d)] ] $}

\sentence{\textit{$\exists$ [\textsc{ag} John] drove fastest} \\
 $\leadsto\exists x \ldot \const{ag}(x,\const{j}) \land
  \const{driving}(x) \land  \exists d [ \const{fast}(x,d) \land \forall x'
  [[C(x') \land x'\neq x] \rightarrow \neg \textsc{fast}(x',d)] ] $
}

%%%%

  
\begin{verbatim}
to interpret a superlative sentence: 

Version 1
first figure out ass'n rel'n via focus;
that gives you c.s. + m.e.; (by abstracting over their positions -
focus for the contrast set, and superlative phrase for the m.e.) 
     BUT if there is no focus, then a.r. defaults to =;
     = can't give you c.s. or m.e.; therefore
           you default to the NOUN's extension (some subset thereof)
           to give you both c.s. + m.e. and you then get an absolute
           reading;
           BUT if you cannot default to the noun b/c there is no noun
           (as is the case w adverbials and predicative),
                THEN you cannot use = as a.r.; 
                AND therefore cannot get an absolute reading

Version 2
first figure out ass'n rel'n via focus;
that gives you c.s. + m.e.; (by abstracting over their positions -
focus for the contrast set, and superlative phrase for the m.e.) 
     BUT if there is no focus, then a.r. defaults to =;
     = can't give you c.s. or m.e.; therefore
           you default to the HEAD's extension (some subset thereof)
           to give you both c.s. + m.e. and you then get an absolute
           reading

~~~~~~~~~~~~~~~~~~~~~~~~~~~~~~~~~~~~~~~~~~~~~~~
Getting a relative reading for John_F drove (the) fastest with the recipe:

John is focused. Abstract over ``drive fastest'' to get other slot.
Association relation = x is agent of (driving event) e.
Therefore c.s. = [[John]]^F
m.e.s = driving event 1, 2, 3... (where agent is among c.s.)
measure (type <d,vt>) = e is d-fast

==...==> There is a driving event e that John is the agent of
and for all distinct elements of the contrast set, 
the driving events that they are the agent of are not as fast as e.

~~~~~~~~~~~~~~~~~~~~~~~~~~~~~~~~~~~~~~~~~~~~~~~
Getting a relative reading for John_F climbed (the) highest mountain with the recipe:

John is focused. Abstract over ``the highest mountain'' to get other slot.
Association relation = x climbed y.
Therefore c.s. = [[John]]^F
m.e.s = climbed mountains 1, 2, 3... (where climber is among c.s.)
measure (type <d,et>) = x is d-tall

==...==> There is a mountain x that John climbed
and for all distinct elements of the contrast set, 
the mountains that they climbed are not as tall as x.

~~~~~~~~~~~~~~~~~~~~~~~~~~~~~~~~~~~~~~~~~~~~~~~
TODO: Show that the recipe doesn't work for absolute readings of
adverbials.

~~~~~~~~~~~~~~~~~~~~~~~~~~~~~~~~~~~~~~~~~~~~~~~
Version 2 absolute reading


Version 2
first figure out ass'n rel'n via focus;
that gives you c.s. + m.e.; (by abstracting over their positions -
focus for the contrast set, and superlative phrase for the m.e.) 
     BUT if there is no focus, then a.r. defaults to =;
     = can't give you c.s. or m.e.; therefore
           you default to the HEAD's extension (some subset thereof)
           to give you both c.s. + m.e. and you then get an absolute
           reading

Step 1: Nothing is focussed so a.r. defaults to =
HEAD's extension = [[drive]] -- a property of events
c.s. and m.e.s are a relevant subset of this. (driving events)
Measure is speed of event.

==...==> There is a driving event x that John is the agent of
and for all distinct elements of the contrast set, 
their speed is not as great as x.

This would be true in a situation with many driving events that John
is the agent of, e.g.

John drove 30mph
John drove 40mph
John drove 50mph

Therefore "John drove fastest".

In this case, the condition that there are multiple entities in the
contrast set is satisfied by the different events with the same agent.

This is bad.
But we can rule it out with a constraint on the contrast set that says
that it can't be too abstract.

Degrees are possibly OK; this could explain upstairs de dicto readings
John's desires are directed at a degree. (measure = identity)

How do we formulate this in a way that is not circular?
Can you have a mass noun in the contrast set?

This water is home to the most sea life.

~~~~~~~~~~~~~~~~~~~~~~~~~~~~~~~~~~~~~~~~~~~~~~~
Version 1
first figure out ass'n rel'n via focus;
that gives you c.s. + m.e.; (by abstracting over their positions -
focus for the contrast set, and superlative phrase for the m.e.) 
     BUT if there is no focus, then a.r. defaults to =;
     = can't give you c.s. or m.e.; therefore
           you default to the NOUN's extension (some subset thereof)
           to give you both c.s. + m.e. and you then get an absolute
           reading;
           BUT if you cannot default to the noun b/c there is no noun
           (as is the case w adverbials and predicative),
                THEN you cannot use = as a.r.; 
                AND therefore cannot get an absolute reading

Step 1: Nothing is focussed so a.r. defaults to =
Therefore you default to the NOUN's extension.
There is no noun. So we have no c.s.
Boom.

~~~~~~~~~~~~~~~~~~~~~~~~~~~~~~~~~~~~~~~~~~~
TOMORROW

Nobody has done anything on superlative adverbs
(Krasikova mentioned it as an outstanding puzzle)

Here's all this interesting data
- the position
- optionality of the determiner
- missing readings:
John drove the fastest =/= John drove the fastest relevant (driving)
event/speed. There have to be other agents, not OK if John's the only
drivers.
ALT: You can't say "John drove the fastest" if there aren't other
people who drove (ignoring reading comparing driving to other types of
events)
vs. you can say "John climbed the highest mountain" if there aren't
other people who climbed mountains.

Comparison with predicative superlatives.
Tell me about Usain Bolt.
Usain Bolt is a 33 year old runner from Jamaica. He is the fastest.


Ingredients: contrast set, assoc reln, measured entities, measure,
domain

Recipe: HEAD version

Ruling out absolute reading: constraints on contrast set

\end{verbatim}

% A proportional reading of an amount superlative would be:

% \sentence{John mostly laughed.}

% \sentence{*John leastly laughed.}

% Consonant with general observation that negative amount superlatives ({\em least}, {\em fewest}) do not have proportional (smallest part of partition) readings.

% \paragraph{Syntactic reststriction}

% \sentence{*John (the) fastest drove.}

% \paragraph{Morphological restrictions}

% {\em lastly}, {\em *prettiestly}, {\em *quickestly}...

% Periphrastic: {\em most elegantly}

\end{document}