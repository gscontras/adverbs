\makeatletter
     
\edef\@backslash{\expandafter\@car\string\relax\@nil}
     
\def\@warnredef#1{\typeout{WARNING: The `#1' control sequence is being
 redefined.}}
     
\def\@warnnoredef#1{\typeout{WARNING: Aborting redefinition of the `#1' control
 sequence.}}
     
\def\@ifcsundefined#1#2#3{\ifx #1\anUndefinedMacro
   \def\@@temp{#2}\else\ifx #1\relax\def\@@temp{#2}\else
   \def\@@temp{#3}\fi\fi\@@temp}
     
% Change LaTex behavior.
     
\def\@notdefinable{\@warnredef{\@backslash\@tempa}}
     
% Define \redefcheck{\ctlseq} that just checks the sequence
% and ignores it.
     
\def\redefcheck#1{\@ifcsundefined#1{}{\@warnredef{\string#1}}}
     
% \redefabort{\oneargdefiner}{\ctlseq} puts out \oneargdefiner\ctlseq,
% except that if \ctlseq is already defined, it just gobbles both
% of them.
     
\def\redefabort#1#2{\@ifcsundefined#2{#1#2}{\@warnnoredef{\string#2}}}
     
% Define \newdef = \def with \redefcheck, and similarly for \gdef and \let.
     
\def\newdef#1{\redefcheck{#1}\def#1}
\def\newgdef#1{\redefcheck{#1}\gdef#1}
\def\newlet#1{\redefcheck{#1}\let#1}
     
% Add redefinition checking to \newif.
% \newif is declared as \outer, so we need to delay the definition
% of \latexnewif until we've scanned the definition of the new
% \newif.
     
\let\@tempa\newif
\def\newif#1{\redefcheck{#1}\latexnewif#1}
\let\latexnewif\@tempa
     
% Now for the allocation commands, which I'm setting up to skip the
% redefinition if their target control sequences are already defined.
% These are also declared \outer, but the above \@tempa trick won't work
% here; it's necessary to use another level of macros.
     
\def\latexnewbox{\latexouternewbox}
\let\latexouternewbox\newbox
\def\newbox#1{\redefabort\latexnewbox{#1}}
     
\def\latexnewcount{\latexouternewcount}
\let\latexouternewcount\newcount
\def\newcount#1{\redefabort\latexnewcount{#1}}
     
\def\latexnewdimen{\latexouternewdimen}
\let\latexouternewdimen\newdimen
\def\newdimen#1{\redefabort\latexnewdimen{#1}}
     
\def\latexnewtoks{\latexouternewtoks}
\let\latexouternewtoks\newtoks
\def\newtoks#1{\redefabort\latexnewtoks{#1}}
     
\makeatother
% end redefine
    
% A referenceable "sentences" list environment for Latex.  Remember to
% put "{}" % after the \item command if the sentence begins with "[".
% There is also a "subsentences" environment.
     
\makeatletter
     
\newif\ifinsentences
\insentencesfalse
     
\newcounter{sentence}
\newcounter{subsentence}[sentence]
     
% These next definitions define the format of the "sentence"
% and "subsentence" counters.  It is tricky to get them right.
% First of all, the parentheses in "(4)" and "(2iii)" are
% defined to be part of the printed representation of the
% "sentence" and "subsentence" counters.  This is so that
% parentheses will be included in \ref - \label constructions.
% Second, the "subsentence" counter must print out with the
% "sentence" number included as well; this is similar to
% what happens with the section and chapter counters.
% Third, despite the second requirement, the "subsentence"
% counter must appear without the sentence number when it
% is actually used to number a subsentence.
%
% We use two auxiliary control sequences here and below in order
% to make it easier to change the subsentence numbering style.
% The two control sequences are \subsentencenumbering, which should
% be \let to something like \roman or \alph, and
% \subsentencepunctuation, which controls the format of subsentence
% numbers as they appear before the subsentences.
     
\def\thesentence{(\arabic{sentence})}
\def\thesubsentence{(\arabic{sentence}\subsentencenumbering{subsentence})}
     
% These next two definitions are defaults.
     
\let\subsentencenumbering\alph
\def\subsentencepunctuation#1{#1.}
     
%%% Using \begin{list} and \end{list} in the argument of a
%%% \newenvironment command doesn't work properly; I have to call
%%% \list and \endlist directly.
     
\newenvironment{sentences}{\list{\thesentence}%
{\edef\sent@nbr{\the\c@sentence}\usecounter{sentence}%
\setcounter{sentence}{\sent@nbr}%
\ifinsentences \errmessage{Nested sentences environment.} \fi
\def\makelabel##1{##1\hfil}%
% the following command prevents indention problems when the number of
% sentences in the document falls in the set {99,...,999}
% (R. E. Wargaski Jr., 31 Oct 89)
\setlength{\leftmargin}{1.0cm}
\leftmargin\leftmargin\labelwidth\leftmargin
  \advance\labelwidth-\labelsep
\insentencestrue}}{\endlist}
     
\newenvironment{subsentences}{\list
{\subsentencepunctuation{\subsentencenumbering{subsentence}}}%
{\edef\sent@nbr{\the\c@subsentence}\usecounter{subsentence}%
\setcounter{subsentence}{\sent@nbr}%
% the following command prevents indention problems when the number of
% sentences in the document falls in the set {99,...,999}
% (R. E. Wargaski Jr., 31 Oct 89)
\setlength{\leftmargin}{1.0cm}
\setlength{\rightmargin}{\leftmargin}
\leftmargin\leftmargin\labelwidth\leftmargin
  \advance\labelwidth-\labelsep
\def\makelabel##1{##1\hfil}%
\ifinsentences\relax\else
   \errmessage{\string\begin{subsentences} not in sentences environment}\fi}}%
{\endlist}
\newcommand{\sentence}[1]{\begin{sentences}\item{#1}\end{sentences}}
     
\newdef\nextsentence{\@ifnextchar[{\@argnextsentence}{\@argnextsentence[1]}%]
}
     
\newdef\@argnextsentence[#1]{\bgroup
  \advance\c@sentence by #1\relax
  \thesentence
  \egroup}
     
\newdef\nextsentencesub{\@ifnextchar[{\@argnextsentencesub
   }{\@argnextsentencesub[1]}%]
}
     
\newdef\@argnextsentencesub[#1]#2{\bgroup
  \advance\c@sentence by #1\relax
  \c@subsentence=#2\relax
  \thesubsentence
  \egroup}

